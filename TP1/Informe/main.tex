\documentclass[10pt,a4paper]{article}
\usepackage[utf8]{inputenc}
\usepackage{caratula}
\usepackage{a4wide}
\usepackage{tikz}
\usepackage{float}
\usepackage{makecell}
\usepackage{tabularx}
\usepackage{placeins}
\usepackage{hyperref}
\hypersetup{
    linktoc=all,     %set to all if you want both sections and subsections linked
    linkcolor=blue,  %choose shttps://www.sharelatex.com/project/542da47afb0e80fe43643f61ome color if you want links to stand out
}
\input{page.layout}

\begin{document}
\titulo{Trabajo Práctico 1}
\subtitulo{\emph{Técnicas Algorítmicas}}

\fecha{\today}

\materia{Problemas, Algoritmos y Programación}

\integrante{Chamo, Nicolás}{282/13}{nicochamo@hotmail.com}
\integrante{Donatucci, Nicolás Andres}{263/13}{nadonatucci@gmail.com}
\integrante{Noriega, Francisco José}{660/12}{frannoriega.92@gmail.com}
\integrante{Zimespitz, Ezequiel}{155/13}{ezeqzim@gmail.com}

\maketitle

\newpage
\thispagestyle{empty}
\vfill
\vspace{3cm}
\tableofcontents
\newpage
\setcounter{page}{1}
\section{Introducción}
\section{Ejercicio 1}
\subsection{El Problema}
\subsection{Desarrollo}
\subsubsection{Implementación}
\subsubsection{Complejidad}
\section{Ejercicio 2}
\subsection{El Problema}
\subsection{Desarrollo}
\subsubsection{Implementación}
\subsubsection{Complejidad}
\section{Ejercicio 3}
\subsection{El Problema}
\subsection{Desarrollo}
\subsubsection{Implementación}
\subsubsection{Complejidad}
\section{Ejercicio 4}
\subsection{El Problema}
\subsection{Desarrollo}
\subsubsection{Implementación}
\subsubsection{Complejidad}

\end{document}
