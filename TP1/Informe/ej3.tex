\section{Ejercicio 3}
\subsection{El Problema}
El problema consiste en encontrar el subarreglo de suma máxima de un arreglo dado. Formalmente, sea $A$ un arreglo de longitud $N$, se quiere encontrar el $i$ y el $j$ tal que se maximice la siguiente función (si no existen $i$ y $j$ tal que la función sea mayor a cero se devuelve 0): 
\begin{center} $\sum_{k=i}^{j} A_k$            $ \quad 0\leq i\leq j\leq N$
\end{center}
Se pide implementar un algoritmo que devuelva el valor máximo de dicha función con complejidad temporal del orden de $O(N)$
\subsection{Desarrollo}
En la solución propuesta, se utilizan dos variables, \texttt{maximo} y \texttt{tmp}. Se va avanzando por el arreglo linealmente y tras el paso $p$, en  \texttt{maximo} se encuentra el valor máximo de la función $\sum_{k=i}^{j} A_k$            $ \quad 0\leq i\leq j\leq p \leq N$ (es decir, la suma de subarreglo máxima del arreglo  \texttt{A[0..p-1]} ) y en  \texttt{tmp} se encuentra la suma de subarreglo máxima tal que contenga al elemento \emph{p-ésimo}, es decir, el valor máximo de  $\sum_{k=i}^{p} A_k$            $ \quad 0\leq i\leq p\leq N$.


Al comenzar el paso $p$,  \texttt{tmp} se actualiza al valor máximo entre  \texttt{A[p-1]} y  \texttt{tmp + A[p-1]}
Al finalizar cada paso, si  \texttt{tmp > maximo}, se actualiza  \texttt{maximo} al valor de  \texttt{tmp}.
De esta forma, al terminar el paso \emph{N-ésimo}, en \texttt{maximo} se va a encontrar la solución al problema.

\subsubsection{Complejidad}
Se analizará la complejidad temporal del algoritmo.
Fuera del \texttt{for} únicamente se realizan operaciones de orden $O(1)$. Dentro del \texttt{for}, en cada iteración se lee un elemento de la entrada y se ejecuta una operación de \texttt{max}, una comparación y si el nuevo valor es mayor al máximo, una asignación. Todas estas operaciones son de orden $O(1)$.
Como el \texttt{for} va de $0$ hasta $N$, la complejidad total del algoritmo es del orden de $O(N)$.

\subsection{Puntaje}
El peso otorgado a este ejercicio es: 9