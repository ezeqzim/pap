\section{Ejercicio 3}
\subsection{El Problema}
El problema consiste en encontrar el subarreglo de suma máxima de un arreglo dado. Formalmente, sea $A$ un arreglo de longitud $N$, se quiere encontrar el $i$ y el $j$ tal que se maximice la siguiente función (si no existen $i$ y $j$ tal que la función sea mayor a cero se devuelve 0): 
\begin{center} $\sum_{k=i}^{j} A_k$            $ \quad 1\leq i\leq j\leq N$
\end{center}
Se asume indexación desde 1 en la explicación para que quede más claro.
Se pide implementar un algoritmo que devuelva el valor máximo de dicha función con complejidad temporal del orden de $O(N)$
\subsection{Desarrollo}
En la solución propuesta, se utilizan dos variables, \texttt{maximo} y \texttt{tmp}. Se va avanzando por el arreglo linealmente y tras el paso $p$, en  \texttt{maximo} se encuentra el valor máximo de la función $\sum_{k=i}^{j} A_k$            $ \quad 1\leq i\leq j\leq p \leq N$ (es decir, la suma de subarreglo máxima del arreglo  \texttt{A[1..p]} ) y en  \texttt{tmp} se encuentra la suma de subarreglo máxima tal que contenga al elemento \emph{p-ésimo}, es decir, el valor máximo de  $\sum_{k=i}^{p} A_k$            $ \quad 1\leq i\leq p\leq N$.


Al comenzar el paso $p$,  \texttt{tmp} se actualiza al valor máximo entre  \texttt{A[p]} y  \texttt{tmp + A[p]}
Al finalizar cada paso, si  \texttt{tmp > maximo}, se actualiza  \texttt{maximo} al valor de  \texttt{tmp}.
De esta forma, al terminar el paso \emph{N-ésimo}, en \texttt{maximo} se va a encontrar la solución al problema.

Más formalmente, podemos ver que inicialmente tanto \texttt{tmp} como \texttt{maximo} comienzan en 0. Esto hace que efectivamente sean la suma máxima del subarreglo vacío y la suma máxima que no contiene a ningún elemento.

Suponemos que vale que, para el paso $p-1$, \texttt{maximo = } $ max \sum_{k=i}^{j} A_k$            $ \quad 1\leq i\leq j\leq p-1 < N$. es decir, \texttt{maximo} contiene la suma de subarreglo máxima del arreglo \texttt{A[1..p-1]} o 0 si no hay elementos positivos. Además suponemos que vale que, para el paso $p-1$, \texttt{tmp = } $\sum_{k=i}^{p-1} A_k$            $ \quad 0\leq i\leq p-1 < N$. Es decir, la suma de subarreglo máxima que si o si contiene al elemento \emph{p-1-ésimo}. Queremos ver que tras el paso $p$ del ciclo, en \texttt{maximo} ahora estará la suma de subarreglo máxima de \texttt{A[1..p]} y en \texttt{tmp} la suma máxima de subarreglo que si o si contiene al elemento \emph{p-ésimo}. 

Sin perdida de generalidad, supongamos que se tiene el arreglo \texttt{[2,3,4,-10,5,4]}. Por lo supuesto, al empezar el paso 5, vale que \texttt{maximo = 9} y \texttt{tmp = -1}.

Primero, el algoritmo actualiza \texttt{tmp} al máximo entre \texttt{A[p]} y  \texttt{tmp + A[p]}. En el ejemplo esto es el máximo entre \texttt{A[5]=5} y \texttt{tmp+A[5] = 4}. Al comenzar el paso nuevo, se está considerando un nuevo arreglo que tiene un nuevo elemento. Es decir, se pasó de \texttt{[2,3,4,-10]} a \texttt{[2,3,4,-10,5]}. Entonces, si se quiere el subarreglo de suma máxima que contenga al 5, éste se puede construir agregándole el 5 al subarreglo de suma máxima que contenga al -10 (\texttt{tmp}), o utilizando \texttt{[5]} como subarreglo. Estas son las dos únicas posibilidades de subarreglo de suma máxima que contenga al 5. En el ejemplo conviene utilizar únicamente el arreglo \texttt{[5]}. Entonces, \texttt{tmp} se actualiza a \texttt{A[5]}, que efectivamente es el valor de la suma del subarreglo de suma máxima que contiene al quinto elemento.

Luego, si \texttt{maximo < tmp}, se actualiza \texttt{maximo} a \texttt{tmp}. En el ejemplo, \texttt{maximo = 9} y \texttt{tmp = 5}. Entonces no se va a actualizar el valor de \texttt{maximo}. Recordamos que \texttt{maximo} contiene la suma de subarreglo máxima del arreglo \texttt{A[1..4] = [2,3,4,-10]}. En el paso actual consideramos el arreglo \texttt{A[1..5] = [2,3,4,-10,5]}. Es decir, si hay una suma de subarreglo máxima en éste arreglo que sea mayor a la del arreglo \texttt{[2,3,4,-10]}, necesariamente va a tener que contener al elemento nuevo, el 5 (dado que si éste subarreglo no incluyera al 5, entonces sería igual al de la iteración anterior). Entonces, se ve que en el arreglo \texttt{[2,3,4,-10,5]}, el valor de la suma de subarreglo máxima es, o bien el valor de la suma de subarreglo máxima de \texttt{ [2,3,4,-10]} (valor en \texttt{maximo}) o la suma de subarreglo máxima que si o si contiene al elemento \texttt{A[5]=5} (valor en \texttt{tmp}). Entonces, por eso, si el valor de \texttt{tmp} es mayor al de \texttt{maximo}, se actualiza \texttt{maximo} al valor de \texttt{tmp} y ahora \texttt{maximo} efectivamente pasa a ser la suma de subarreglo máxima del arreglo \texttt{A[1..p]}.

Entonces, por lo visto recién, al finalizar el paso $N$, en \texttt{maximo} se tiene el valor máximo de $\sum_{k=i}^{j} A_k$            $ \quad 1\leq i\leq j\leq N$, que es la solución al problema.

\subsubsection{Complejidad}
Se analizará la complejidad temporal del algoritmo.
Fuera del \texttt{for} únicamente se realizan operaciones de orden $O(1)$. Dentro del \texttt{for}, en cada iteración se lee un elemento de la entrada y se ejecuta una operación de \texttt{max}, una comparación y si el nuevo valor es mayor al máximo, una asignación. Todas estas operaciones son de orden $O(1)$.
Como el \texttt{for} va de $0$ hasta $N-1$, la complejidad total del algoritmo es del orden de $O(N)$.

\subsection{Puntaje}
El peso otorgado a este ejercicio es: 9