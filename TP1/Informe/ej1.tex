\section{Ejercicio 1}
\subsection{El Problema}
El problema toma como entrada un límite $P$ y $N$ valores $a_{i}$, y busca la suma de los elementos del conjunto máximo de valores cuya suma sea la máxima posible, y que sea menor a $P$.\\
Formalmente, \\
\begin{center}
Sea $A = \{a_{1},...,a_{n}\}$ y sea $P$ el límite mencionado, y sea $A' \subset A / \displaystyle\sum_{a \in A'} a < P \wedge (\forall B \subset A) B \neq A \wedge \displaystyle\sum_{b \in B} b < P \Rightarrow \displaystyle\sum_{a \in A'} a > \displaystyle\sum_{b \in B} b$, \\se busca el valor $L = \displaystyle\sum_{a \in A'} a$
\end{center}
Este problema es un caso particular del \emph{Problema de la Mochila\textsuperscript{\cite{knapp}}}, en donde el peso de cada elemento se corresponde al valor del mismo.

\subsection{Desarrollo}
La solución consta de dividir los $N$ valores entrantes en dos, y para cada grupo de valores (que llamaremos $A$ y $B$), calcula el conjunto de partes, sólo que en lugar de guardar los subconjuntos, guarda la suma de los elementos de los mismos. Agrega también el valor $0$ a ambos grupos.\\
Luego, ordena ambos grupos de menor a mayor, de manera tal que $A = [a_{1},...,a_{\vert A\vert}]$ con $a_{i} < a_{i+1}$, y $B = [b_{1},...,b_{\vert B\vert}]$ con $b_{i} < b_{i+1}$.\par
Finalmente, realiza una busqueda para encontrar un $a \in A$ y un $b \in B$ tales que $a + b \leq P$ y $(\forall x \in A)(\forall y \in B) x \neq a \wedge y \neq b \Rightarrow x + y < a + b$. Dichos valores siempre existen dado que $P$ es un entero positivo y $a$ y $b$ podrían eventualmente ser $0$., y para hallarlos, recorre los elementos de $A$ de izquierda a derecha, y los de $B$ de derecha izquierda, y calcula si la suma $a_{i} + b_{j} > P$. En todo momento, se mantiene una variable (que llamaremos $MAX$) que recuerda cual fue el valor de la máxima suma encontrada.\\
Si esa suma es mayor que $P$, avanza $j$, es decir, repite el proceso con $a_{i} + b_{j-1}$. Ya que $B$ esta ordenado de menor a mayor, si $a_{i} + b_{j} > P \Rightarrow (\forall k > j) a_{i}+b_{k} > P$, por lo que todos los $b_{k}$ ya no son necesarios.\\
Si esa suma es menor que $P$, se actualiza $MAX$ con el valor de dicha suma de ser necesario, y se avanza $i$, es decir, repite el proceso con $a_{i+1} + b_{j}$. Ya que $A$ esta ordenado de menor a mayor, si $a_{i} + b_{j} < P \Rightarrow (\forall k < i) a_{k}+b_{j} < P$, por lo que todos los $a_{k}$ ya no son necesarios.\par
Al final del proceso, se retorna el valor $MAX$.
\subsubsection{Implementación}
\begin{verbatim}
void generateParts(vector<ll>& v, int num, int iter){
  //La posicion donde agregar el nuevo elemento
  //(o sea, desde 0..size esta todo lo que fui calculando)
  int size = (1<<iter)-1;
  v[size] = num;
  //Recorre todo lo que ya calcule
  forn(i, size){
    //Y lo guarda en las posiciones siguientes
    v[size+i+1] = num+v[i];
  }
}

int getMax(vector<ll>& first, vector<ll>& second, int limit){
  int max = 0, sum;
  int j = second.size()-1;
  forn(i, first.size()){
    while(first[i] + second[j] > limit && j > 0) j--;
    sum = first[i]+second[j];
    if(sum > limit) return max;
    if(sum > max) max = sum;
  }
  return max;
}

int main() {
  int P, N;
  cin >> P >> N;
  int half = N/2;
  //Guarda 2^(n/2) en cada arreglo, para guardar las partes
  vector<ll> firstHalf((1<<half)+1);
  vector<ll> secondHalf((1<<(N-half))+1);
  int num;
  forn(i, N){
    cin >> num;
    if(i < half)
      //Pasa el i como parametro para saber en que iteracion esta
      generateParts(firstHalf, num, i);
    else
      generateParts(secondHalf, num, i-half);
  }
  firstHalf[1<<half] = 0;
  secondHalf[1<<(N-half)] = 0;
  sort(firstHalf.begin(), firstHalf.end());
  sort(secondHalf.begin(), secondHalf.end());
  cout << getMax(firstHalf, secondHalf, P) << endl;
  return 0;
}
\end{verbatim}

\subsubsection{Complejidad}
Estudiaremos la complejidad de las tres funciones presentadas: \emph{generateParts, getMax} y \emph{main}:
\begin{itemize}
	\item \textbf{generateParts}: Actualiza el conjunto de partes, agregando el valor $num$, para lo cual recorre todos los subconjuntos previamente generados. Su complejidad es $O(2^n)$.
	\item \textbf{getMax}: Recorre dos arreglos $A$ y $B$ y va sumando los valores. Recorre ambos una única vez. Su complejidad es $O(A+B)$.
	\item \textbf{main}: Esta función realiza las siguientes operaciones:
	\begin{enumerate}
		\item Crea dos arreglos de tamaño $O(2^{N/2})$. $O(2^{N/2})$.
		\item Por cada elemento, llama a \textbf{generateParts}. $O(N*2^{N/2})$
		\item Ordena los dos arreglos. $O(log_{2}(2^{N/2})) = O(N/2)$
		\item Llama a la función \textbf{getMax} con ambos arreglos. $O(2^{N/2} + 2^{N/2}) = O(2^{N/2})$.
	\end{enumerate}
	Por la suma de dichas operaciones y por propiedades de la \emph{Cota superior}, esta función tiene complejidad $O(N*2^{N/2})$
\end{itemize}