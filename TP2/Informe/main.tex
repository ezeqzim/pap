\documentclass[10pt,a4paper]{article}
\usepackage[utf8]{inputenc}
\usepackage{caratula}
\usepackage{geometry}
\usepackage{mathtools}
\usepackage{algpseudocode}
\usepackage{algorithm}
\usepackage{tikz}
\usepackage{float}
%\usepackage{makecell}
\usepackage{tabularx}
\usepackage[nomove]{cite}
\usepackage[nottoc]{tocbibind}
\usepackage{placeins}
\usepackage{hyperref}
\hypersetup{
    linktoc=all,     %set to all if you want both sections and subsections linked
    linkcolor=blue,  %choose shttps://www.sharelatex.com/project/542da47afb0e80fe43643f61ome color if you want links to stand out
}
\renewcommand{\O}[1]{$\mathcal{O}(#1)$}
\newcommand{\incfig}[2]{\begin{figure}[H]\centering\includegraphics[width=15cm]{#1}\caption{#2}\end{figure}}
\input{page.layout}

\begin{document}
\titulo{Trabajo Práctico 2}
\subtitulo{\emph{Grafos}}

\fecha{\today}

\materia{Problemas, Algoritmos y Programación}

\hashGrupo{Emmy Noether}

\maketitle

\newpage
\thispagestyle{empty}
\vfill
\vspace{3cm}
\tableofcontents
\newpage
\setcounter{page}{1}

\section{Ejercicio 1}
\subsection{El Problema}

Obtenemos un polígono simple de $N$ lados, descompuesto en $N - 2$ triángulos. Los vértices de los mismos son también vértices del polígono original, tal que dos triángulos cualesquiera no tienen un punto estrictamente interior en común, y de manera tal que la unión de todos los triángulos produce el polígono original. 

Lo que debemos hacer es reconstruir el polígono original, devolviendo  los $N$ vértices del mismo ordenados en sentido horario. Además debemos hacer en \O{N log N}.
\subsection{Desarrollo}
\subsubsection{Inspiración divina}

Lo primero que hicimos fue darnos cuenta de lo siguiente:
\textit{Todo par de vértices puede aparecer cero, una o dos veces como lado de un triángulo. Si aparece una, ese par es un lado del polígono original. Si aparece dos, no lo es.}

Veamos por qué esto es cierto. 

Un par de vértices puede no aparecer como lado de un triángulo. Esto se debe a que la descomposición en triángulos no es única, por lo que en la descomposición provista un par de vértices no se encuentra unido.
* Imagen de rectángulo *


Explicar xq aparece una
Explicar xq podría aparecer 2.
Explicar xq no 3 o más.
Explicar xq si aparece una es externo y si aparece 2 es interno

\subsubsection{Solución}
Nuestra idea entonces es ir guardando los distintos lados de los triángulos, pero si encontramos uno ya guardado, lo borramos en vez de guardarlo. Entonces, solo nos quedaríamos con los que aparecen una vez, es decir, los del polígono original.

\underline{\textbf{Implementación}}
Cuando leemos un triángulo,  ABC, para A guardamos o borramos B y C, para B A y C, y para C, A y B. Así se vería:
*Imagen*

Una vez leídos todos los triángulos, nos quedaría lo siguiente:
*Imagen*

Después, tomamos el punto más abajo a la izq (*Revisar*) y elegimos el punto adyacente que respete el sentido horario (*Revisar*). Desde ahí, es cuestión de ir recorriendo los puntos sin repetir hasta volver al inicial.
\subsubsection{Complejidad}


\subsection{Puntaje}
El peso otorgado a este ejercicio es:

\newpage
\section{Ejercicio 2}
\subsection{El Problema}

Dado un arreglo $A$ de $N$ números, se desea saber la cantidad de pasos esperada para ordenarlo, utilizando el siguiente algoritmo de ordenamiento:
\begin{enumerate}
	\item Se comienza con una variable $i = 0$ y el arreglo $A$ de tamaño $N$ como entrada.
	\item Se permuta al azar el arreglo $A[i..N)$.
	\item Mientras $i < N$ y $\min(A[i..N)) = A[i]$, incrementar $i$ en $1$.
	\item Si $i < N$, volver al punto 2.
\end{enumerate}

El problema consiste en encontrar la esperanza de la cantidad de permutaciones que deberán realizarse hasta ordenar el arreglo. Se pide implementar un algoritmo que encuentre este número con una complejidad del orden de \O{N^2}.

\subsection{Desarrollo}
Para resolver el problema, primero se definen:
\begin{itemize}
	\item $MIN_{i}$ como el elemento de mínimo valor de $A[i..N)$.
	\item $CM_i$ como la cantidad de veces que $MIN_i$ aparece en $A[i..N)$.
	\item $P_i(X)$ como la probabilidad de que, estando en el paso \emph{i-ésimo} (es decir, cuando $A[0..i)$ ya esta ordenado), después de una permutación, $i$ avance hasta que $i = X$.
	\item $Q(i)$ como la probabilidad de tomar $MIN_i$ de $A[i..N)$. Notar que $Q(N) = 0$.
\end{itemize}

\noindent Consideremos que:
\begin{itemize}
	\item Como $i$ siempre avanza, la probabilidad de volver a un estado anterior es $0$.
	\item Si el algoritmo se encuentra en el paso \emph{N-ésimo}, entonces $i$ no puede avanzar más. Luego, la probabilidad de que $i$ no avance es $1$.
	\item La probabilidad de avanzar $i$ en $1$, es igual a la probabilidad de encontrar $MIN_i$ de entre todos los elementos que componen $A[i..N)$. Como $MIN_i$ se puede repetir en $A[i..N)$, la probabilidad de avanzar $i$ en $1$ es igual a la $Q(i)$ por la probabilidad de no tomar ningun $MIN_{i+1}$ (es decir, avanzo $i$ exactamente en $1$, $(1-Q(i+1)$).
	\item La probabilidad de avanzar $i$ hasta un valor $X$ se define recursivamente, como la probabilidad de avanzar $i$ en $1$, por la probabilidad de avanzar desde $i+1$ hasta el valor $X$.
\end{itemize}

\noindent Consecuentemente, esto define a $P_i(X)$ de la siguiente forma:
\[
	P_i(X) = 
		\begin{cases}
			0 & \quad \text{Si }X < i \\
			1 - Q(i) & \quad \text{Si }X = i \\
			Q(i) * P_{i+1}(X) & \quad \text{En otro caso}
		\end{cases}
\]


Luego, se define $M \in \mathbb{R}^{NxN}$ la matriz de probabilidades, donde $M_{i,j} = P_i(j)$.
Dicha matriz cumple la propiedad de ser una matriz estocástica derecha (es decir, que sus filas estan conformadas por valores reales no negativos cuya suma es $1$) y triangular superior.\\
También posee la propiedad de que $M_{i,i} = 1 - P_i(i+1)$ (es decir, la probabilidad de que la permutación no permite avanza el $i$).\\

Entonces, se define la funcion $h(t)$ como la cantidad de pasos necesaria para llegar al estado final (es decir, al arreglo ordenado), desde la posición actual $t$ (es decir, el momento en el que $i = t$). Por lo tanto, $t = 1,...,N$, y
\[ 
	h(t) = 
	\begin{cases}
		\displaystyle 1 + \sum_{i = t}^{N} P_i(N)*h(i) & \quad \text{Si }t < N \\
		0 & \quad \text{En otro caso}
	\end{cases}
\]

\noindent Se definen $E \in \mathbb{R}^N$ y $U \in \mathbb{R}^N$ como 
\[ E = (h(1),...,h(N))\text{ y }U = (1,...,1) \]

\noindent Para encontrar los valores de $h(t)$ para $t = 1,...,N$, solo basta resolver el sistema $M*E + U = E$.\\

\noindent Finalmente, el resultado del algoritmo es el valor $h(1)$, es decir, $E_1$.

\subsubsection{Complejidad}
El algoritmo para resolver el problema consta de cuatro funciones importantes:
\begin{itemize}
	\item \textbf{calcularMins}: Función que calcula el vector $minimos$, donde $minimos[i] = $ Cantidad de mínimos en $A[i..N)$. Esta función recorre el arreglo $A$ una única vez, y posee complejidad \O{N}.
	\item \textbf{calcularPs}: Función que calcula el vector $PS$, donde $PS[i] = Q(i)$. Esta función recorre el arreglo $A$ una única vez, y posee complejidad \O{N}.
	\item \textbf{generarProbabilidades}: Función que calcula la matriz $M$. \\
	Dado que $M$ es triangular superior, se calcularon las posiciones $M_{i,j}$ donde $j \geq i$. \\
	Sabiendo que $M_{i,j} = P_i(j)$, luego se puede reemplazar por \[ M_{i,j} = Q(i) * P_{i+1}(j) = PS[i] * M_{i+1,j} \]
	Por esta propiedad, es posible recorrer solo la porción triangular superior de $M$, y recorriendola desde la última fila y última columna, hasta llegar a la primer fila, se pueden calcular las posiciones usando los valores previamente calculados, como se vé en la fórmula.\\
	Consecuentemente, la complejidad es \O{N^2}.
	\item \textbf{obtenerEsperanza}: Función encargada de calcular el vector $E$. Al comienzo, $E$ es un vector de $0$'s. \\
	Dado que se conoce el valor $h(N)$ y que $M$ es triangular superior, nuevamente la mejor estrategia es recorrer solo la porción triangular superior de $M$, recorriendola desde la última fila y última columna, hasta llegar a la primer fila.\\
	Por cada fila, se calcula $\displaystyle E_i = 1 + \sum_{t = i}^{N} M_{i,t}*E_{t}$. Esto es equivalente a resolver el sistema de ecuaciones utilizando la técnica de \emph{Backwards Substitution}\textsuperscript{\cite{backsust}}.\\
	Finalmente, la complejidad es \O{N^2}.
\end{itemize}

\noindent Por la suma de dichas operaciones y por propiedades de la \emph{Cota superior}, la complejidad total es \O{N^2}.

\subsection{Puntaje}
El peso otorgado a este ejercicio es:

\newpage
\section{Ejercicio 3}

\subsection{El Problema}

\subsection{Desarrollo}

\subsubsection{Complejidad}

\newpage
\section{Ejercicio 4}
\subsection{El Problema}
Se tienen $A$ aulas y $P$ pasillos, que se pueden recorrer unidireccionalmente. Se desea saber, a través de una serie de $Q$ consultas, si es posible llegar desde un aula $A_{i}$ a un aula $A_{j}$ y volver al aula $A_{i}$.

\subsection{Desarrollo}
Para resolver el problema en cuestion, se modeló un grafo dirigido $G$ $=$ $(V,E)$ con $V$ $=$ $A$ y $E$ $=$ $P$.
En dicho grafo, se calcularon las componentes fuertemente conexas.
Por definición de componente fuertemente conexa, si dos aulas $A_{i}$ y $A_{j}$ pertenecen a la misma componente, eso significa que es posible llegar desde $A_{i}$ a $A_{j}$ y volver a $A_{i}$\textsuperscript{\cite{cfc}}.
Por lo tanto, las consultas se responden simplemente verificando si las aulas por las que se pregunta, pertenecen a la misma componente.

\subsubsection{Complejidad}
Nuestro algoritmo consta de las siguientes partes
\begin{itemize}
	\item Lectura de la entrada y armado del grafo: Se crean los vértices $A$ y se van agregando los $P$ pasillos. Complejidad: \O{A+P}.
	\item Calculo de componentes fuertemente conexas: Para el calculo de las componentes, se utilizó el algoritmo de Kosaraju, que calcula las componentes fuertemente conexas, y cuya complejidad es \O{V+E}\textsuperscript{\cite{cormen}}. La implementación de dicho algoritmo se corresponde con la indicada en el libro \emph{Introductions to Algorithms}\textsuperscript{\cite{cormen}}, con la única diferencia de que cuando se armó el grafo $G$, se armó tambien el grafo $G^T$ (es decir, el calculo de $G^T$ no se lleva a cabo en la ejecución de Kosaraju). Complejidad: \O{A+P}.
	\item Consultas: Las consultas se realizan de forma inmediata, simplemente preguntando si dos aulas pertenecen a la misma componente conexa. Complejidad: \O{Q}.
\end{itemize}

Por propiedad de la \emph{cota asintótica superior}, la complejidad total es \O{A+P+Q}.
\newpage
\section{Apéndice}
\subsection{Edmonds Karp}\label{edmkarp}
\begin{algorithmic}
	\State Inicializar flujo en 0
	\While{HayCaminoDeAumento}
		\State Incrementar flujo
		\State ActualizarRedResidual
	\EndWhile
	\State Retornar flujo
\end{algorithmic}
\vspace{1em}
\begin{algorithmic}
\Function{HayCaminoDeAumento}{}
	\State Hacer DFS pasando solo por nodos no visitados y que aun no hayan llenado el flujo
\EndFunction

\Function{ActualizarRedResidual}{}
	\For{eje en camino de aumento}
		\State Saturar eje
		\State Agregar eje en direccion opuesta
	\EndFor
\EndFunction
\end{algorithmic}

\emph{NOTA: Se consideraró que todos los ejes tienen capacidad 1. Por lo tanto, de haber camino de aumento, su flujo será 1. Agregar ejes en dirección opuesta supone que en peor caso se hagan el doble de DFS}

\bibliographystyle{unsrt}
\bibliography{main}
\end{document}
