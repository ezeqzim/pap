\documentclass[10pt,a4paper]{article}
\usepackage[utf8]{inputenc}
\usepackage{caratula}
\usepackage{geometry}
\usepackage{mathtools}
\usepackage{algpseudocode}
\usepackage{algorithm}
\usepackage{tikz}
\usepackage{float}
%\usepackage{makecell}
\usepackage{tabularx}
\usepackage[nomove]{cite}
\usepackage[nottoc]{tocbibind}
\usepackage{placeins}
\usepackage{hyperref}
\hypersetup{
    linktoc=all,     %set to all if you want both sections and subsections linked
    linkcolor=blue,  %choose shttps://www.sharelatex.com/project/542da47afb0e80fe43643f61ome color if you want links to stand out
}
\renewcommand{\O}[1]{$\mathcal{O}(#1)$}
\newcommand{\incfig}[2]{\begin{figure}[H]\centering\includegraphics[width=15cm]{#1}\caption{#2}\end{figure}}
\input{page.layout}

\begin{document}
\titulo{Trabajo Práctico 2}
\subtitulo{\emph{Grafos}}

\fecha{\today}

\materia{Problemas, Algoritmos y Programación}

\hashGrupo{Emmy Noether}

\maketitle

\newpage
\thispagestyle{empty}
\vfill
\vspace{3cm}
\tableofcontents
\newpage
\setcounter{page}{1}

\newtheorem*{definition}{Definición}
\newtheorem{lem}{Lema}
\section{Ejercicio 1}
\subsection{El Problema}
Dado dos cadenas de caracteres $W$ y $P$, se desea saber si $P$ es una subcadena de $W$.\\
Formalmente, se define $W = [w_1,...,w_N]$ y $P = [p_1,...,p_k]$, y se desea saber si existe $W_P = [w_q,...,w_t] = P$, con $1 \leq q \leq t \leq N$.\\
El algoritmo debe tener complejidad \O{|W|}.

\subsection{Desarrollo}
Para resolver el problema, se utilizó el algoritmo de \emph{Knuth-Morris-Pratt}.
Por lo tanto, la solución se divide en dos partes:
\begin{itemize}
	\item Cálculo de bordes de la cadena $P$
	\item Búsqueda de $W_P$
\end{itemize}

\subsubsection{Cálculo de bordes de la cadena $P$}
Esta sección explica unicamente cómo se realiza el calculo de bordes y por qué se realiza correctamente. En la sección siguiente se explica como se utiliza para resolver el problema.

Primero, veamos algunas definiciones y propiedades:
\begin{definition}
$P_k \sqsubset P_i \Leftrightarrow P_k$ es prefijo de $P_i$. Analogamente vale $P_k \sqsupset P_i \Leftrightarrow P_k$ es sufijo de $P_i$.
\end{definition}
\begin{definition}
Se denomina borde de una cadena $S$ a una subcadena que es prefijo y sufijo de $S$. Es decir, $Q$ es borde de $S \Leftrightarrow Q \sqsubset S \wedge Q \sqsupset S$
\end{definition}

\begin{lem}
La relación $\sqsubset$ es transitiva.
\end{lem}
\begin{proof}
Sean $A = [a_1,...,a_k]$, $B = [b_1,...,b_n]$, $C = [c_1,...,c_m]$, cadenas tales que $A \sqsubset B$, $B \sqsubset C$, queremos ver si $A \sqsubset C$.\\
Como $A \sqsubset B$, luego $A = [b_1,...,b_k]$, y como $B \sqsubset C$, luego $B = [c_1,...,c_n]$. Entonces $A = [c_1,...,c_k]$ y por lo tanto $A \sqsubset C$.
\end{proof}

\begin{lem}
Si $T$ es borde de $S$ y $Q$ es borde de $T$, entonces $Q$ es borde de $S$.
\end{lem}
\begin{proof}
Por \textbf{Lema 1}, al ser $Q \sqsubset T$ y $T \sqsubset S$, luego $Q \sqsubset S$. Análogamente sucede con los sufijos.
\end{proof}

\begin{lem}
Si $T$ y $Q$ son bordes de $S$ y $|Q|<|T|$, entonces $Q$ es borde de $T$.
\end{lem}
\begin{proof}
Sea $T = [t_1,...,t_N]$, $Q = [q_1,...,q_M]$ y $S = [s_1,...,s_O]$ con $M < N < O$, como $Q \sqsubset S$ y $T \sqsubset S$, luego $T = [s_1,..., s_{M-1},s_M,s_{M+1},...,s_N]$ y $Q = [s_1,...,s_M]$. Por lo tanto $Q \sqsubset T$. Análogamente sucede con los sufijos.
\end{proof}

\begin{lem}
Si $T$ y $Q$ son bordes de $S$ y $Q$ es el mayor borde de $T$, entonces $Q$ es el mayor borde de $S$ de longitud menor a $|T|$.
\end{lem}
\begin{proof}
Supongamos que $Q$ no fuera el mayor borde de $S$ de longitud menor a $|T|$, luego existiria un $Q'$ que fuese el mayor borde de $S$ de longitud menor a $|T|$. Ya que $Q'$ es un borde de $S$, eso significa que $Q' \sqsubset S$, y como $|Q'| < |T|$ entonces por el \textbf{Lema 3}, $Q'$ sería el borde de $T$, y al ser $|Q'| > |Q|$, $Q'$ sería el mayor borde de $T$. Absurdo, pues $Q$ era el mayor borde de $T$.
\end{proof}

Por comodidad, en adelante llamaremos $P_k$ al prefijo de $P$ de tamaño $k$, y llamaremos $\Pi$ a la estructura que posee las siguientes propiedades:
\begin{itemize}
	\item $\pi_i = max \{k: k < i \wedge P_k \sqsupset P\} = $ Longitud del mayor borde de $P$ de longitud menor a $i$.
	\item $\Pi = [\pi_1,...,\pi_{|P|}]$
\end{itemize}

\textbf{Correctitud}\\
A continuación se adjunta un pseudocódigo para facilitar la comprensión de la demostración de correctitud:\\
\begin{algorithm}[H]
 \DontPrintSemicolon
 \KwData{$P$ cadena a la que calculo los bordes, 
 $\Pi$ arreglo vacío de longitud $|P|$, 
 $j$ tamaño de la cadena máxima encontrada}
 \KwResult{$\Pi$}
 $\Pi[0] \gets 0$\\
 $j \gets 0$\\
 \For{$i \gets 1...|P|-1$}{
		\While{$j > 0 \wedge P[i] \neq P[j]$}{
			$j \gets \Pi[j-1]$
		}
		\uIf{$P[i] = P[j]$}{
			$j \gets j+1$
		}
		$\Pi[i] = j$
 }
 \caption{Cálculo de $\Pi$}
\end{algorithm}

En las líneas 1-3, $\Pi[0]$ es $0$ ya que el borde de mayor tamaño del prefijo de tamaño $1$ es $0$ (es decir, la cadena vacía). Luego, se itera la cadena $P$ desde $1$ ya que no es necesario ver el caso del prefijo de tamaño $1$.\\
Al comienzo de cada iteración, $j$ expresa el tamaño del mayor borde de $P_{i-1} \sqsubset P$, $|P_{i-1}| = i-1$.
En las líneas 4-6, y por el \textbf{Lema 4}, lo que se intenta es extender el borde de mayor tamaño de entre todos los $P_{k}$, $k < i$, tal que sea borde de $P_i$.\\
Si es posible extender el borde de longitud $j$, lo extiendo (líneas 7-8).\\
En la última línea, agrego en $\Pi$ la longitud del borde de mayor tamaño que pude encontrar, es decir, el borde de mayor tamaño de $P_i$ (puede ser $0$).

Al finalizar, se tendrá un arreglo $\Pi$ que cumple las propiedades previamente mencionadas.\\
\strut\hfill\qedsymbol

\textbf{Complejidad}\\
Dado que el \textbf{for} realiza $|P|$ iteraciones, sólo demostraremos que las operaciones realizadas en el cuerpo del \textbf{for} tienen complejidad \O{1} amortizado, demostrando así que el algoritmo tiene complejidad \O{|P|}.
Veremos que el cuerpo del \textbf{while} se ejecuta como mucho $|P|-2$ veces a lo largo de toda la ejecución del \textbf{for}, pero antes, algunas observaciones sobre $j$.\\
Primero, la línea 2 comienza con $j$ en $0$, y la única forma de incrementar $j$ es en la línea 8, donde se incrementa en $1$, que se ejecuta a lo sumo una vez por cada iteración del \textbf{for}.\\
Segundo, dado que $j < i$ al comienzo del \textbf{for} y que en cada iteración se incrementa el valor de $i$ en $1$, entonces la desigualdad $j < i$ se mantiene a lo largo de toda la ejecución del algoritmo. Por lo tanto, las asignaciónes en las líneas 1 y 9 garantizan que $\pi_i < i$ para todo $i = 1,2,...,|P|-1$, lo que significa que en cada iteración del \textbf{while}, se decrementa el valor de $j$.\\
Tercero, $j$ nunca es negativo.\\
Luego, con todos esos datos, se puede ver que el valor total por el que se puede decrementar $j$ durante la ejecución del \textbf{while} esta acotado por el valor máximo por el que se puede incrementar $j$, que es $|P|-2$, a lo largo de todas las iteraciones del \textbf{for}.
En consecuencia, el \textbf{while} itera a lo sumo $|P|-2$ veces a lo largo de toda la ejecución del \textbf{for}, por lo que su costo amortizado es \O{1}.

Luego, el algoritmo tiene complejidad \O{|P|}.\\
\strut\hfill\qedsymbol

\subsubsection{Búsqueda de $W_P$}
A continuación se adjunta el algoritmo:\\
\begin{algorithm}[H]
 \DontPrintSemicolon
 \KwData{$P$, $W$, 
 $\Pi$ arreglo calculado en la sección anterior, 
 $j$ tamaño de la subcadena de $W$ más grande que es prefijo de $P$}
 \KwResult{Devuelve $true$ si existe $W_P$. De lo contrario devuelve $false$}
 $j \gets 0$\\
 \For{$i \gets 0...|W|-1$}{
		\While{$j > 0 \wedge W[i] \neq P[j]$}{
			$j \gets \Pi[j-1]$
		}
		\If{$W[j] = P[i]$}{
			$j \gets j+1$
		}
		\uIf{$j = |P|$}{
			Devolver $true$
		}
 }
 Devolver $false$
 \caption{Búsqueda de $W_P$}
\end{algorithm}

Esencialmente, el algoritmo es igual al de cálculo de bordes, con la siguiente diferencia de que en las líneas 3-5, no se estan extendiendo bordes, sino que si el caracter que se esta observando no coincide, en lugar de comenzar de cero, se comienza desde el borde.\\

Luego, si los caracteres son iguales, el tamaño de la subcadena de $W$ más grande que es prefijo de $P$ encontrado hasta el momento se incrementa.
Finalmente, si $j = |P|$, significa que encontró $W_P$, y devuelvo $true$. Si el \textbf{for} termina de ejecutar, significa que no existe $W_P$, y devuelve $false$.

\textbf{Complejidad}\\
Por el mismo razonamiento que en el algoritmo de \emph{cálculo de bordes de $P$}, se puede observar que el \textbf{while} se ejecuta $|W|-1$ veces a lo largo de toda la ejecución del algoritmo, con lo que el \textbf{while} tiene un costo amortizado de \O{1}. Luego, como el \textbf{for} itera $|W|$ veces, el algoritmo tiene complejidad \O{|W|}.

\subsection{Complejidad}
La complejidad del algoritmo es la suma de las complejidades de cada etapa:
\begin{itemize}
	\item Cálculo de bordes de la cadena $P$: \O{|P|}
	\item Búsqueda de $W_P$: \O{|W|}
\end{itemize}

Por la suma de dichas operaciones y por propiedades de la \emph{Cota superior}, como $|P| \leq |W|$, la complejidad total es \O{|W| + |P|} $ = $ \O{2*|W|} $=$ \O{|W|}.

\subsection{Puntaje}
El peso otorgado a este ejercicio es: 10

\newpage
\section{Ejercicio 2}

\subsection{El Problema}

El problema toma como entrada un conjunto P de $N$ personas y una matriz M de $N \times N$, tal que $M_{ij}$ almacena qu\'e tan bien se llevan $P_i$ con $P_j$.

Un valor negativo implica que se llevan mal, cero que les es indistinto y positivo que se llevan bien. Vale que $M_{ii} = 0$ $\forall i = 1,...,N$ y $M_{ij} = M_{ji}$  $\forall i,j = 1,...,N$. Es decir que M es sim\'etrica y con diagonal nula. 

Interesa construir una partici\'on de $P = \biguplus\limits_{k = 1}^{K} A_k, K \leq N$ tal que se maximice $\displaystyle\sum_{k = 1}^{K} valor(A_k)$.

El valor de un conjunto es igual a la suma de qu\'e tan bien se llevan cada par de personas dentro del mismo.

\subsection{Desarrollo}

Representaremos a los conjuntos con un entero \textit{mask}, cuya representaci\'on en binario indica que si el bit i-\'esimo es un $1$ entonces $P_i \in mask_i$.

La soluci\'on consiste en calcular, para cada elemento del conjunto de partes de P, la suma de los $M_{ij}$ correspondientes. Luego buscamos la partici\'on que cumpla el requisito de la secci\'on anterior.

La funci\'on funScore, dado un conjunto A, calcula el m\'aximo entre:

\begin{itemize}
	\item $0$ que significa que tomamos $\#A$ subconjuntos de $A$.
	\item valor(A).
	\item ($\forall A_i \subset A$) score($A_i$) + score($A_i^c$) que significa que tomamos entre $2$ y $\#A-1$ subconjuntos de A.
\end{itemize}

\subsubsection{Complejidad}

En este apartado analizaremos la complejidad de las funciones \textit{main}, \textit{funScore} y \textit{setToScore}

\begin{itemize}
	\item \textbf{setToScore}: La funci\'on hace un for de 0 a N que realiza otro for de i+1 a N. Las dem\'as operaciones son $O(1)$. Esto da como resultado una complejidad de $O(N^2)$.

	\item \textbf{funScore}: La funci\'on calcula el score de mask $O(N^2)$, luego itera sobre los subconjuntos calculando la suma de su score con el del complemento en forma recursiva, y se queda con el m\'aximo entre ellos.
	Cabe destacar que para cada subconjunto posible, se llama solo una vez a la funci\'on \texttt{setToScore}, dado que se memoizan los resultados. En total son $O(2^N)$ subconjuntos, sumado a la iteraci\'on sobre los subconjuntos de \textit{mask}, se puede demostrar que la complejidad final es de $O(3^N)$.

	\item \textbf{main}: La funci\'on lee la entrada, el entero $N$, luego $N^2$ enteros, inicializa el memo \textit{fun}, que tiene $O(2^N)$ posiciones. Luego llama a la funci\'on funScore $O(3^N)$. La complejidad es entonces $O(1 + N^2 + 2^N + 3^N)$ que es igual a $O(3^N)$.
\end{itemize}

\newpage
\section{Ejercicio 3}
\subsection{El Problema}
En este problema interesa construir una muralla para salvar lugares históricos.

Por un lado, nos brindan H puntos donde se encuentran estos lugares históricos que nos gustaría salvar. Por el otro, nos dicen la ubicación de E edificios tomados por enemigos que no queremos mantener dentro de la nueva muralla.

Nos piden un algoritmo polinomial (preferentemente \O{(H+E)^4}) que compute el polígono convexo que contenga a la mayor cantidad de edificios históricos, pero que no contenga ningún edificio tomado por enemigos. En particular interesa la cantidad máxima de edificios que se pueden salvar y no el polígono en sí.

Desde ahora llamaremos puntos \textit{buenos} a los lugares históricos a salvar y \textit{malos} a los que pertenecen a los edificios tomados por enemigos.

\subsection{Estrategia}
Para resolver este problema, planteamos el siguiente esquema:

Si supiéramos cuál es el punto que se encuentra más abajo a la izquierda del polígono solución, entonces podemos calcular la cápsula convexa pedida con este punto como referencia. Este punto será llamado \textit{\textbf{pivot}}.

Notemos que si el polígono solución tiene $k > 2$ vértices, puede ser dividido en $k - 2$ triángulos con vértices \textit{pivot}-$punto_i$-$punto_{i+1}$. Vamos a querer conocer cuantos puntos buenos y malos tiene cada uno de estos triángulos, para usar esta información más tarde.

Por ejemplo, dada la siguiente imagen, donde los puntos verdes son los buenos y el rojo el único malo.

\ig{Imagenes/ej3/Imagen_A.jpg}{Posible input del problema}

Puede formarse un triángulo que contiene un punto bueno adentro. Y puede formarse uno que contiene un punto malo adentro.

\miniig{Imagenes/ej3/Imagen_B.jpg}{Posible triángulo a formar con puntos buenos dentro}{Imagenes/ej3/Imagen_C.jpg}{Posible triángulo a formar con puntos malos dentro}

Supongamos ahora que ya encontramos el punto \textit{pivot}, y que calculamos los triángulos y sus contenidos. Veamos ahora cómo construir el polígono buscado:

Si H = 0, no hay polígono que construir, si H = 1, ese punto será la cápsula máxima, si H = 2, entonces el segmento entre ellos será solución, pues no existen 3 puntos colineales.

Veamos el caso $H \geq 3$:

En primer medida vamos a ordenar los H puntos respecto al ángulo con el \textit{pivot}, en sentido anti-horario.

\ig{Imagenes/ej3/Imagen_D.jpg}{Puntos ordenados por ángulo respecto al pivot, en sentido anti-horario}

Además, como no existen 3 puntos colineales, cualquier segmento desde \textit{pivot} hacia los demás puntos son candidatos a solución.

La idea será la siguiente: para cada punto $X$ distinto al \textit{pivot}, armar todos los triángulos posibles con otros puntos buenos anteriores, que llamaremos $Y$. Los siguientes son los triángulos posibles desde el punto 4:

\ig{Imagenes/ej3/Imagen_E.jpg}{Triángulos posibles desde el punto 4}

Si uno de estos triángulos tiene dentro un punto malo, lo descartamos. Si no es así, calculamos recursivamente los posibles triángulos desde $Y$ que además sean convexos respecto al triángulo \textit{pivot}-$X$-$Y$ y no tengan dentro puntos malos. Si seguimos tomando a $X$ como 4 y tomamos a $Y$ como 3, estos serían los posibles sub triángulos.

\ig{Imagenes/ej3/Imagen_F.jpg}{Posibles sub triángulos con X = 4 e Y = 3}

Cabe aclarar que ninguno de estos triángulos forma una figura convexa con el triángulo P-4-3, por lo que no serán tenidos en cuenta para la solución final.

Ahora, en cada uno de estos pasos, podemos ir contando la cantidad de cantidad de puntos buenos que quedan dentro del polígono que estamos armando. De esta forma vamos a encontrar el mejor polígono posible para el \textit{pivot} elegido. Lo único que falta ahora es repetir el algoritmo eligiendo a un \textit{pivot} nuevo en cada iteración y devolver el máximo entre todas las iteraciones.

\subsection{Implementación}
En primera instancia precomputamos todas las triangulaciones posibles desde cada punto. Para cada uno de estos, guardamos además cuántos puntos buenos y cuántos malos tienen dentro cada uno. Por motivos de eficiencia, al tener que reordenar el arreglo de puntos buenos de acuerdo al \textit{pivot} en cada iteración, guardamos en este el índice por el cual indexar el precómputo de los triángulos.

La forma que decidimos implementar el algoritmo descrito previamente fue la siguiente: a partir de un punto cualquiera (distinto al \textit{pivot}), recorrer todos los que están entre este y el \textit{pivot}. Recordemos que estos ya se encuentran ordenados.

De esta manera llamaremos:
\begin{itemize}
  \item \textit{current} = cada uno de los puntos desde el tercero en adelante. \textcolor{blue}{yo le pongo tercero porque tengo en cuenta el pivot} \textcolor{red}{Aclaralo entonces}

  \item \textit{prevLast} = cada uno de los puntos entre \textit{pivot} y current - 1.

  \item \textit{last} = cada uno de los puntos entre prevLast + 1 y current.
\end{itemize}

Si los tres puntos se encuentran en sentido anti-horario respeto a \textit{last} \textcolor{red}{(Esto me hace ruido, como está escrito)} \textcolor{blue}{es una mierda de escribir :)}, entonces mantienen el atributo de convexidad buscado, por lo tanto, esa configuración es un candidato válido.

Dadas estas definiciones, se derivan los triángulos con vértices \textit{pivot}-\textit{last}-\textit{current} (PLC) y \textit{pivot}-\textit{prevLast}-\textit{last} (PPL).

Si en PLC hay algún punto \textit{malo}, no nos sirve y lo descartamos. Si no contiene ningún punto \textit{malo}, entonces una posible solución será igual a la cantidad de puntos \textit{buenos} dentro de PLC + 1 (agregar \textit{current} al polígono) + la mejor solución hallada en PPL (este caso es recursivo). Nuestro caso base es siempre es 2, es decir el \textit{pivot} + cualquier otro punto.

Notamos también que tenemos subproblemas repetidos, que no queremos recalcular cada vez, con lo cual, podemos guardarlos en un \textbf{memo}. Las claves de este serán los puntos \textit{last} y \textit{current} e indicarán la máxima cantidad de puntos que se salvan usando el triángulo PLC para toda posible configuración.

Alcanza entonces con devolver el máximo de estos cálculos.

Finalmente, tenemos que eliminar el \textit{pivot} de los puntos buenos y repetir el proceso para contemplar todos los casos, pues partimos de la hipótesis de que el \textit{pivot} pertenece al polígono solución.

\subsubsection{Complejidad}
Como mencionamos previamente, lo primero que hacemos es calcular todos los triángulos que se forman con los puntos buenos. En total son \O{N^3} triangulos (elijo 1 vértice de los N, otro de los N-1 restantes y uno más de los N-2 finales).

Por cada triángulo, calculamos la cantidad de puntos buenos y malos dentro de él. Es decir que para cada triángulo ABC, por cada punto P creamos 4 segmentos AB, AC, BC, PK (\O{1}) y calculamos la cantidad de intersecciones entre ellos. Si la cantidad de intersecciones es par, entonces el punto está afuera, si no, está adentro. K es un punto que calculamos al principio que está fuera de todo posible triángulo \O{N}. La intersección de segmentos son simples cuentas, y por lo tanto \O{1}. En definitiva el costo de esta etapa es \O{N + N^3 * N} = \O{N^4}.

Una vez calculado esto, aplicamos el algoritmo descrito previamente. Tenemos \O{H} $\in$ \O{N} puntos \textit{buenos} sobre los que iterar. Por cada uno de estos, ordenamos los puntos (buscar el \textit{pivot} y ordenar según el ángulo, \O{N + N log(N)}), luego iterar los índices \textit{current} entre 2 y H (\O{N}), \textit{prevLast} entre 0 y current (\O{N}) y \textit{last} entre prevLast + 1 y current (\O{N}), esto es \O{N^3}.

Entonces por cada punto \textit{bueno}, tardamos \O{N log(N) + N^3}. Como tenemos \O{N} puntos buenos que iterar, la complejidad de esta parte es \O{N^4}.

Finalmente, el problema se resuelve en 2 etapas de costo \O{N^4 + N^4}, lo que se resume en \O{N^4} como se pedía.

\subsection{Puntaje}
El peso otorgado a este ejercicio es:

\newpage
\section{Ejercicio 4}
\subsection{El Problema}
\subsection{Desarrollo}
\subsubsection{Implementación}
\subsubsection{Complejidad}

\newpage
\section{Apéndice}
\subsection{Edmonds Karp}\label{edmkarp}
\begin{algorithmic}
	\State Inicializar flujo en 0
	\While{HayCaminoDeAumento}
		\State Incrementar flujo
		\State ActualizarRedResidual
	\EndWhile
	\State Retornar flujo
\end{algorithmic}
\vspace{1em}
\begin{algorithmic}
\Function{HayCaminoDeAumento}{}
	\State Hacer DFS pasando solo por nodos no visitados y que aun no hayan llenado el flujo
\EndFunction

\Function{ActualizarRedResidual}{}
	\For{eje en camino de aumento}
		\State Saturar eje
		\State Agregar eje en direccion opuesta
	\EndFor
\EndFunction
\end{algorithmic}

\emph{NOTA: Se consideraró que todos los ejes tienen capacidad 1. Por lo tanto, de haber camino de aumento, su flujo será 1. Agregar ejes en dirección opuesta supone que en peor caso se hagan el doble de DFS}

\bibliographystyle{unsrt}
\bibliography{main}
\end{document}
