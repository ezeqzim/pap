\section{Apéndice}
\subsection{Edmonds Karp}\label{edmkarp}
\begin{algorithmic}
	\State Inicializar flujo en 0
	\While{HayCaminoDeAumento}
		\State Incrementar flujo
		\State ActualizarRedResidual
	\EndWhile
	\State Retornar flujo
\end{algorithmic}

\vspace{1em}

\begin{algorithmic}
\Function{HayCaminoDeAumento}{}
	\State Hacer DFS pasando solo por nodos no visitados y que aun no hayan llenado el flujo
\EndFunction

\vspace{1em}

\Function{ActualizarRedResidual}{}
	\For{eje en camino de aumento}
		\State Saturar eje
		\State Agregar eje en direccion opuesta
	\EndFor
\EndFunction
\end{algorithmic}

\emph{NOTA 1: Se consideró que todos los ejes tienen capacidad 1. Por lo tanto, de haber camino de aumento, su flujo será siempre 1.}

\emph{NOTA 2: "Agregar eje en direcci\'on opuesta" a lo sumo duplica la cantidad de ejes en la red residual \footnote{Por la nota 1, los ejes de un camino de aumento no pueden ser elegidos en otro.}. Esto implica que encontrar un camino de aumento no modifica la cota de complejidad y sigue siendo \O{E}. Como el flujo se puede aumentar a lo sumo \O{VE}\textsuperscript{\cite{cormen}} veces, la complejidad del algoritmo es \O{VE^2}}
